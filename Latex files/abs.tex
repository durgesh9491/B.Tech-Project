\newpage
\quad
\newpage
\baselineskip=18pt
\addcontentsline{toc}{chapter}{ABSTRACT}
\begin{center}
{\large \bf ABSTRACT}
\end{center}

In our work, we have developed a compartmental model using firewall security coefficient for the analysis of the spread of a distributed attack on critically targeted groups in a network. The model gives an epidemic design with two sub-designs to consider the difference between the overall behavior of the attacker class and the targeted class. The targeted nodes and attacker nodes  are divided into five compartments as Susceptible - Latent - Breaking out - Recovered- Antidotal. The boundedness of the system, the feasibility of equilibrium states and their stabilities are analyzed using cyber mass action incidence. Basic reproduction number ${{\cal R}}_{0}$ is observed and it is found that when ${{\cal R}}_{0}<1$, then the system will possess malicious code free stable steady state and, when ${{\cal R}}_{0}>1$, then endemic steady state exists and will have local asymptotic stablity. The impact of firewall security rule base in controlling transmission of malicious objects is analyzed. Some researchers explored the impact of media awareness in biological disease spread using mathematical modeling with transmission coefficient function $\beta(I)=\beta e^{-m\frac{I}{N}}$ and observed that many positive equilibria are possible when the media effect is adequately strong among population \cite{cui2008impact,liu2008impact,sahu2012,sahu2015dynamics}. Similarly we are taking firewall security as a media coverage factor in our computer network model of malicious code propagation. The coefficient of firewall security 'm' depends on the types of files under consideration, defined firewall security rules in the firewall rule base and the reliability and efficiency of the firewall. We suggest a way for quantifying the coefficient m of firewall security as :
          $$m=-{\log_e (a+b-ab)},$$
where, 'b' measures the response of the files to the defined security rules. It is considered that the malware propagation rate can be reduced by a proportion 'a' when all received files abide the defined security rules.
The stability of the system is observed using local asymptotic stability method. Finally, most sensitive system parameters for basic reproduction number are observed using normalized forward sensitivity index. Numerical simulation has been carried out to verify analytical findings.


{\it Keywords:} Computer networks, Compartmental model, Basic reproduction number, Local asymptotic stability, Numerical simulation, Sensitivity analysis, Firewall security rule base. 